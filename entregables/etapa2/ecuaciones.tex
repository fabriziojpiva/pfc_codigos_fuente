\documentclass[12pt]{article}
\usepackage[spanish]{babel}
\usepackage[utf8x]{inputenc}
\usepackage{amsmath}
\usepackage{graphicx}
\usepackage[colorinlistoftodos]{todonotes}
\usepackage{float} %Este paquete es para que se usen los entornos de las tablas, de modo que se puedan fijar en alguna posici�n en particular.
\usepackage{hyperref} %Este paquete es para poder insertar hiperv�nculos en el documento, como por ejemplo direcciones de e-mail.
\usepackage{textcomp} %Paquete para poner los logotipos de registrado,copyright, etc.
\usepackage{listings} %Paquete para los c�digos en cualquier lenguage.
\usepackage{amsfonts}
\usepackage{amssymb}
\usepackage{eufrak}
                      
\begin{document}

\begin{equation}
\label{eq:combinacionlinealtrixy}  
  \vec{v} = 
  N_{1}(\vec{p}) \cdot \vec{v_1} + 
  N_{2}(\vec{p}) \cdot \vec{v_2} + 
  N_{3}(\vec{p}) \cdot \vec{v_3} 
\end{equation}  

\begin{equation}
\label{eq:combinacionlinealcuaxy}  
  \vec{v} = 
  N_{1}(\vec{p}) \cdot \vec{v_1} + 
  N_{2}(\vec{p}) \cdot \vec{v_2} + 
  N_{3}(\vec{p}) \cdot \vec{v_3} +  
  N_{4}(\vec{p}) \cdot \vec{v_4} 
\end{equation}  

\begin{equation}
  \label{eq:verticesmastertri}
  \vec{P_1} \rightarrow 
  \begin{bmatrix}
    0\\ 
    0
  \end{bmatrix}\quad \vec{P_2} \rightarrow
  \begin{bmatrix}
    1\\
    0
  \end{bmatrix}\quad \vec{P_3} \rightarrow
  \begin{bmatrix}
    0\\
    1
  \end{bmatrix}
\end{equation}

\begin{equation}
  \label{eq:chietadexy}
  \begin{split}
    \xi(x,y) & =
    \frac{x(y_3-y_1)+y(x_1-x_3)+y_1x_3-y_3x_1}{(x_2-x_1)(y_3-y_1)+(y_1-y_2)(x_3-x_1)} \\
    \eta(x,y) & =
    \frac{x(y_1-y_2)+y(x_2-x_1)+y_2x_1-y_1x_2}{(x_2-x_1)(y_3-y_1)+(y_1-y_2)(x_3-x_1)}
  \end{split}
\end{equation}

\begin{equation}
  \label{eq:fftridechieta}
  \begin{split}
    N_1(\xi,\eta) & = 1 - \xi - \eta \\
    N_2(\xi,\eta) & = \xi            \\
    N_3(\xi,\eta) & = \eta           
  \end{split}
\end{equation}

\begin{equation}
  \label{eq:combinacionlinealtrichieta}  
  \begin{split}
    \vec{v} & = 
    N_{1}(\xi,\eta) \cdot \vec{v_1} + N_{2}(\xi,\eta) \cdot \vec{v_2}
    + N_{3}(\xi,\eta) \cdot \vec{v_3} \\
    & = (1-\xi-\eta) \cdot \vec{v_1} + \xi\vec{v_2} + \eta\vec{v_3}
    \end{split}
\end{equation}  

\begin{equation}
  \label{eq:mapeoinversocuadrado}
  \begin{split}
    x & = a_1 + a_2\xi + a_3\eta + a_4\xi\eta \\
    y & = b_1 + b_2\xi + b_3\eta + b_4\xi\eta
  \end{split}
\end{equation}

\begin{equation}
  \label{eq:ffcuadechieta}
  \begin{split}
    N_1 & = \frac{ (1-\eta)(1-\xi) }{4} \\
    N_2 & = \frac{ (1+\eta)(1-\xi) }{4} \\
    N_3 & = \frac{ (1+\eta)(1+\xi) }{4} \\
    N_4 & = \frac{ (1-\eta)(1+\xi) }{4} 
  \end{split}
\end{equation}

\begin{equation}
  \label{eq:pdechieta}
  \vec{P} = \frac{1}{4}(\vec{P_1} + \vec{P_2} + \vec{P_3} +
  \vec{P_4}) + \frac{\xi}{4}(\vec{P_2} - \vec{P_1} + \vec{P_3} -
  \vec{P_4}) + \frac{\eta}{4}(\vec{P_3} - \vec{P_2} + \vec{P_4} -
  \vec{P_1}) + \frac{\xi\eta}{4}(\vec{P_1} - \vec{P_2} + \vec{P_3} -
  \vec{P_4}) 
\end{equation}

\begin{equation}
\label{eq:vdechieta}  
  \vec{v} = 
  \frac{ (1-\eta)(1-\xi) }{4} \cdot \vec{v_1} + 
  \frac{ (1+\eta)(1-\xi) }{4} \cdot \vec{v_2} + 
  \frac{ (1+\eta)(1+\xi) }{4} \cdot \vec{v_3} +  
  \frac{ (1-\eta)(1+\xi) }{4} \cdot \vec{v_4} 
\end{equation}  

\begin{equation}
  \label{eq:interpolacionlineal}
  \vec{P} = (1-\alpha)\vec{P_1} + \alpha\vec{P_2} \qquad \alpha\in[0,1]
\end{equation}

\begin{equation}
  \label{eq:xnode}
  \begin{bmatrix}
    x_1 & y_1 & z_1 \\
    x_2 & y_2 & z_2 \\
    \vdots & \vdots & \vdots \\
    x_n & y_n & z_n
  \end{bmatrix}
\end{equation}

\begin{equation}
  \label{eq:icone}
  \begin{bmatrix}
    n_{11} & \cdots & n_{1N} \\
    \vdots & \ddots & \vdots \\
    n_{M1} & \cdots & n_{MN}
  \end{bmatrix}
\end{equation}

\begin{equation}
  \label{eq:estados}
  \begin{bmatrix}
    n'_{1} & \vec{e}_{1} \\
    \vdots & \vdots \\
    n'_{MN} & \vec{e}_{MN}
  \end{bmatrix}
  \qquad \vec{e}_i\in\mathbb{R}^k,\,1\leq i\leq MN,\,k\in \mathbb{N}
\end{equation}


\end{document}

%%% Local Variables:
%%% TeX-master: informe.tex
%%% End: